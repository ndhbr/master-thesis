\section{Einleitung}\label{einleitung}
\subsection{Einleitung und Motivation}\label{einleitung-und-motivation}
Die Wahl des richtigen Studiengangs ist ein wichtiger Schritt im Leben eines jeden angehenden Studierenden. Sie legt den Grundstein für die akademische und berufliche Entwicklung und beeinflusst den individuellen Bildungsweg maßgeblich. Angesichts der Vielfalt an verfügbaren Studiengängen stehen Studieninteressierte vor einer komplexen Entscheidung, die durch eine breite Palette von Inhalten,  Schwerpunkten und Karrierewegen geprägt ist. Diese Vielzahl an Informationen kann zu Unsicherheit und Verwirrung führen, wodurch nicht selten falsche Studienentscheidungen getroffen werden. \parencite{beckmann_verbesserung_2021}

Die vorliegende Masterarbeit greift dieses weit verbreitete Problem auf und stellt einen innovativen Ansatz zur Studienorientierung vor. Das Ziel ist es, eine benutzerfreundliche, automatisch generierte Infografik-basierte Plattform zu entwickeln, die Studieninteressierte dabei unterstützt, fundierte Entscheidungen über ihren zukünftigen Bildungsweg zu treffen. Die Anwendung namens StudyMap kombiniert moderne Datenanalyse, interaktive Benutzeroberflächen und nutzerzentriertes Design, um den Orientierungsprozess zu optimieren.

\subsection{Problemstellung/Zielsetzung}\label{problemstellung-zielsetzung}
Die derzeitige Studienorientierung wird oft durch eine Informationsflut und eine begrenzte Transparenz der verfügbaren Studiengänge behindert. Häufig kennen Studieninteressierte einen Studiengang oder eine ungefähre Richtung, in die sie gehen möchten. Um ähnliche Studiengänge oder alle Studiengänge in der gewünschten Richtung zu finden, fehlt oft der Überblick. \parencite{beckmann_verbesserung_2021} Diese Unsicherheit kann zu falschen Studienentscheidungen führen, die wiederum Studienabbrüche durch Motivationsmangel zur Folge haben \parencite{heublein_ursachen_2010}. Das Hauptziel dieser Masterarbeit ist es, ein innovatives Tool zu entwickeln, das diese Herausforderungen adressiert und die Studienorientierung für Studieninteressierte optimiert.

Die spezifischen Ziele dieser Arbeit sind:
\begin{enumerate}
\item Entwicklung eines Konzepts für einen Studiengangsfinder auf Basis einer
interaktiven Infografik-basierten Benutzeroberfläche.
\item Implementierung und technische Umsetzung des entwickelten Systems am
Beispiel der Ostbayerischen Technischen Hochschule Regensburg (OTH-Regensburg).
\item Evaluation und Validierung eines Prototypen durch Tests mit potenziellen
Nutzern und Analyse der Ergebnisse.
\item Diskussion der Stärken und Schwächen des entwickelten Systems sowie der
möglichen Auswirkungen auf die Studienberatung.
\item Die Ergebnisse dieser Arbeit sollen dazu beitragen, die
Studienorientierung für Studieninteressierte zu erleichtern und ihnen bessere
Informationen und Orientierungshilfen zur Verfügung zu stellen.
\end{enumerate}

Die folgenden Abschnitte dieser Arbeit beschäftigen sich mit dem theoretischen
Hintergrund, der angewandten Methodik, der Konzeption und Umsetzung des
Studiengangsfinders an der OTH-Regensburg, der Evaluierung der Plattform sowie
einer Diskussion und einem Ausblick auf zukünftige Entwicklungen und
Anwendungen.