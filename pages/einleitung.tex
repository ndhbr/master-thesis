\section{Einleitung}\label{einleitung}
\subsection{Einleitung und Motivation}\label{einleitung-und-motivation}
Die Wahl des richtigen Studiengangs ist ein wichtiger Schritt im Leben eines jeden angehenden Studierenden. Sie legt den Grundstein für die akademische und berufliche Entwicklung und beeinflusst den individuellen Bildungsweg maßgeblich. Angesichts der Vielfalt an verfügbaren Studiengängen stehen Studieninteressierte vor einer komplexen Entscheidung, die durch eine breite Palette von Inhalten,  Schwerpunkten und Karrierewegen geprägt ist. Diese Vielzahl an Informationen kann zu Unsicherheit und Verwirrung führen, wodurch nicht selten falsche Studienentscheidungen getroffen werden. \parencite{beckmann_verbesserung_2021}

Die vorliegende Masterarbeit greift dieses weit verbreitete Problem auf und stellt einen innovativen Ansatz zur Studienorientierung vor. Das Ziel ist es, eine benutzerfreundliche, automatisch generierte Infografik-basierte Plattform zu entwickeln, die Studieninteressierte dabei unterstützt, fundierte Entscheidungen über ihren zukünftigen Bildungsweg zu treffen. Die Anwendung namens StudyMap kombiniert moderne Datenanalyse, interaktive Benutzeroberflächen und nutzerzentriertes Design, um den Orientierungsprozess zu optimieren.

\subsection{Problemstellung/Zielsetzung}\label{problemstellung-zielsetzung}
Die derzeitige Studienorientierung wird oft durch eine Informationsflut und eine begrenzte Transparenz der verfügbaren Studiengänge behindert. Häufig kennen Studieninteressierte einen Studiengang oder eine ungefähre Richtung, in die sie gehen möchten. Um ähnliche Studiengänge oder alle Studiengänge in der gewünschten Richtung zu finden, fehlt oft der Überblick. \parencite{beckmann_verbesserung_2021} Diese Unsicherheit kann zu falschen Studienentscheidungen führen, die wiederum Studienabbrüche durch Motivationsmangel zur Folge haben \parencite{heublein_ursachen_2010}.

\noindent
Die spezifischen Ziele dieser Arbeit sind:
\begin{enumerate}
\item Entwicklung eines Konzepts für einen Studiengangsfinder auf Basis einer
interaktiven Infografik-basierten Benutzeroberfläche.
\item Implementierung und technische Umsetzung des entwickelten Systems am
Beispiel der Ostbayerischen Technischen Hochschule Regensburg (OTH-Regensburg).
\item Evaluation und Validierung eines Prototypen durch Tests mit potenziellen
Nutzern und Analyse der Ergebnisse.
\item Diskussion der Stärken und Schwächen des entwickelten Systems sowie der
möglichen Auswirkungen auf die Studienberatung.
\item Die Ergebnisse dieser Arbeit sollen dazu beitragen, die
Studienorientierung für Studieninteressierte zu erleichtern und ihnen bessere
Informationen und Orientierungshilfen zur Verfügung zu stellen.
\end{enumerate}

\subsection{Gliederung der Arbeit}
Die Arbeit beginnt nach der Einleitung mit dem theoretischen Hintergrund. Es werden bereits existierende Werkzeuge und Konzepte zur Studienorientierung beschrieben und die Theorie der Anwendung von interaktiven Systemen definiert.

Anschließend folgt das Kapitel \nameref{sec:methodik}, in dem zunächst verschiedene mögliche Algorithmen zur Darstellung aller Studiengänge besprochen werden. Daran anknüpfend werden die Datenquellen für den Studiengangsfinder hinsichtlich ihrer Herkunft bzw. Beschaffung diskutiert.

Das nächste Kapitel beschreibt das Konzept des innovativen Studiengangsfinders. Zunächst wird die Hochschule Regensburg als Fallbeispiel vorgestellt, einschließlich ihrer Herausforderungen und Ziele. Im Anschluss daran wird ein Konzept entwickelt, wie das Aussehen des Studiengangsfinders sein könnte und wie die Interaktivität gewährleistet werden kann. Außerdem wird in diesem Kapitel die Datenpflege und -sicherung behandelt. Dabei wird explizit auf die Datensicherung, die Trennung von Produktiv- und Testumgebung sowie auf eine Administrationsoberfläche eingegangen.

Das fünfte Kapitel behandelt den User-centered Design Prozess. Anhand von zwei Studien wird das Konzept evaluiert und verfeinert. Die erste Studie ist eine Mockup-Studie, in der das Designkonzept des Studiengangsfinders diskutiert wird. Auf der Grundlage dieser Ergebnisse wird ein Prototyp entwickelt, mit dem eine Prototypstudie mit 40 Personen, die sich für ein Studium interessieren, durchgeführt und diskutiert wird.

Nach Abschluss der Studien wird im Kapitel \nameref{sec:implementierung-und-deployment} zuerst die Auswahl der Technologien besprochen und daraufhin die Herausforderungen, die während der Implementierung entstanden sind. Schließlich werden die erläuterten Einzelkomponenten zusammengeführt und die daraus resultierende Softwarearchitektur, unterteilt in Frontend und Backend, erläutert. Der letzte Abschnitt dieses Kapitels befasst sich mit dem Software-Deployment. Es wird gezeigt, wie das Deployment abläuft, wie getestet werden kann und wie die Software letztendlich bereitgestellt wird.

Abschließend werden im letzten Kapitel die Ergebnisse und die wichtigsten Erkenntnisse zusammengefasst. Darüber hinaus wird ein Vergleich zu anderen Tools im Bereich der Studienorientierung gezogen und ein Ausblick gegeben. Im Rahmen dieses Ausblicks werden mögliche Erweiterungen und Anwendungen für die Zukunft diskutiert.