\section{Studie: Test an fachfremden Studierenden}\label{studie}
\subsection{Allgemein}
In einer Studie sollen Fehler und schwierig gestaltete Stellen der
Programmierplattform gefunden und analysiert werden. Damit die Studie möglichst
realitätsnah ist, fällt die Wahl hierbei auf eine Feldstudie mit fachfremden
Studierenden.

Nach Vollendung der Tests sollen Schwachstellen und Lücken der
Programmieraufgaben (Anleitungen, Fehlerbeschreibungen, Aufbau,
Schwierigkeit, ...) gefunden, behoben und für zukünftige Aufgaben berücksichtigt
werden. \parencite{studie-testkonzept}

\subsection{Testaufbau}
Um für jeden Durchlauf gleiche Testbedingungen sicherzustellen, wird ein
Testkonzept ausgearbeitet. Bei jeder Durchführung wird sich auf die Regeln
und den Ablauf des Testkonzepts bezogen.

Das Testkonzept legt im Grunde die Meta-Daten der Studie fest. Neben der
Zielsetzung, der Zielgruppe und der Dauer des Tests wird in dem Konzept auch die
Methodik festgelegt.

\subsubsection{Zielgruppe}
Die Zielgruppe der Studie lässt sich durch wenige Parameter definieren. Sie
ist äquivalent zur Zielgruppe des Zusatzstudiums Digital Skills. Gesucht sind
folglich Studierende, welche keine Studiengänge der Fakultät Informatik und
Mathematik belegen. Durch diese Einschränkung qualifiziert man sich für die
Teilnahme am Zusatzstudium Digital Skills und dadurch implizit auch für die
Anteilnahme an der folgenden Studie.

Um eine gleichmäßig verteilte Stichprobenmenge zu erhalten, werden mindestens
fünf Testpersonen aus fünf unterschiedlichen Studiengängen für die Durchführung
benötigt.

\subsubsection{Dauer des Tests}
% TODO: Quelle Feldstudie?
Die Dauer des Tests wird zwar als groben Leitfaden auf 30 bis 60 Minuten
festgelegt, soll jedoch keine harten Limits festlegen. Der Test soll möglichst
naturgemäß ablaufen und kann bei möglichen Unverständnissen auch deutlich länger
dauern. Gemäß einer Feldstudie darf das Ergebnis der Studie nicht durch eine
mögliche Manipulation, durch beispielsweise einer zeitlichen Barriere,
verfälscht werden.

\subsubsection{Methodik}
Die Durchführung der Tests geschieht zur Vereinfachung online über eine Software
für Videokonferenzen. Am Anfang wird vom Beobachter der Durchführung der
Studie, Andreas Huber, das Zusatzstudium Digital Skills anhand einer
einseitigen Infografik vorgestellt.

Danach wird dem Teilnehmer die folgende Agenda vorgetragen und anschließend
werden der Person vier Fragen gestellt. Folgende Fragen müssen mit Schulnoten
von 1 bis 6 beantwortet werden. Sechs bedeutet hier immer eine gänzliche
negative Neigung, während eins einer völligen Zusage entspricht. Die restlichen
Noten bilden wie gewohnt Zwischenwerte.
\begin{itemize}
    \item Hast du schon einmal erwägt eine Programmierausbildung anzustreben?
    \item Hast du schon einmal programmiert?
    \item Wie fit fühlst du dich am PC?
    \item Würde für dich das Zusatzstudium Digital Skills in Frage kommen?
\end{itemize}

Nachdem die Fragen von der teilnehmenden Person beantwortet wurden, muss diese
die Bildschirmübertragung starten. Währendessen schreibt der Beobachter die
Ergebnisse der Fragen in eine vorher angelegte Tabelle.

% TODO: Quelle Think-Aloud-Methodik
Der Teilnehmer wird anschließend darauf hingewiesen, dass die Studie der
Think-Aloud-Methodik folgt. Die Think-Aloud-Methode ist eine Forschungsmethode,
bei der der Testteilnehmer darum gebeten wird, seine Gedanken laut
auszusprechen. Mit dieser Methode können mögliche Aufhänger und Probleme in den
Anleitungen der Aufgaben leichter gefunden werden. Der Beobachter des Tests
schreibt alle Anomalien kategorisiert nach Fortschritt und Teilnehmer des Tests
auf.

Aufkommende Fragen werden können vom Beobachter beantwortet werden. Sollte eine
Frage bzw. Aufgabe unlösbar erscheinen, gibt es am Ende jeder Aufgabe ein
Lösungsvorschlag. Dieser sollte jedoch nur aufgeklappt werden, wenn keine
realistische Chance der selbstständigen Lösung des Problems besteht.

Folgend werden die im folgenden Kapitel \ref{studie-aufgaben} erläuterten
Aufgaben durch den Testteilnehmer bearbeitet. Während der Bearbeitung wird
die dafür benötigte Zeit durch den Beobachter mit einer Stoppuhr gestoppt.

Abschließend kann der Studierende die Bildschirmübertragung wieder beenden und
sich auf das Nachgespräch konzentrieren. Hierbei werden dem Teilnehmer noch
folgende fünf Fragen gestellt. Die Fragen müssen wie vorher in Schulnoten von 1
bis 6 beantwortet werden.
\begin{itemize}
    \item Wie schwer kam dir die Einrichtung, bis zu dem Punkt, an dem du die erste Aufgabe heruntergeladen hast, vor?
    \item Hattest du Schwierigkeiten mit Aufgabe 1?
    \item Hattest du Schwierigkeiten mit Aufgabe 2?
    \item Hast du verstanden, was du in den Aufgaben genau gemacht hast?
    \item Würdest du nach Abschluss des Tests deine Meinung zur Frage am
    Interesse eines Zusatzstudiums für Digital Skills ändern?
\end{itemize}

Die Ergebnisse der Fragen werden, wie gewohnt, vom Beobachter in der Tabelle
notiert.

\subsubsection{Die Aufgaben}\label{studie-aufgaben}
Der Test besteht aus vier für die Studie vereinfachten Aufgaben. Die erste
Aufgabe beschäftigt sich lediglich mit der Einrichtung des Arbeitsplatzes in
Replit und GitHub-Classroom. Die Studierenden sind dazu aufgefordert, das in 
Kapitel \ref{replit-template} besprochene Replit-Template zu klonen und in 
Replit einzurichten. Dazu gehört das Hinzufügen des SSH-Keys zu GitHub, sowie
das Herunterladen der ersten Aufgabe.

Die zweite Aufgabe ist eine Programmieraufgabe mit der Sprache Python. Es
handelt sich hierbei um eine vereinfachte Version der später in produktiv
eingesetzten Aufgabe \glqq Lab 4: Hello\grqq{}. Die Vorlage der Aufgabe
frägt den Nutzer nach seinem Namen. Nach der Eingabe schließt sich das Programm
wieder. Die Aufgabe des Teilnehmers ist nun den Namen in folgendem Format
wieder auszugeben: \code{Hallo, mein Name ist <NAME>}. Diese Aufgabe kann der
Teilnehmer mit einer einzigen Codezeile lösen. Die Ausgabe der Lösung sieht dann
so aus:

\begin{lstlisting}
    Wie ist dein Name? Andreas
    Hallo, mein Name ist Andreas!
\end{lstlisting}

Der Name ist wie vorher beschrieben flexibel und abhängig von der Eingabe des
Nutzers.

Die dritte Aufgabe ist ebenfalls eine Programmieraufgabe mit der
Sprache Python. In dieser Aufgabe geht es um die erste Verwendung einer
\texttt{for}-Schleife. Am Anfang der Anleitung wird sehr ausführlich erklärt,
was eine \texttt{for}-Schleife ist, weshalb man sie braucht und wie man sie
anwendet. Danach wird die Aufgabenstellung erklärt, welche ähnlich zur zweiten
Aufgabe ist. Dieses Mal wird der Nutzer nach Start des Programms nicht nur nach
seinem Namen gefragt, sondern auch nach der Anzahl, wie oft der Name ausgegeben
werden soll. Wie auch in der vorherigen Aufgabe schließt sich standardmäßig das
Python-Skript gleich wieder. Der Testteilnehmer muss nun eine Schleife
programmieren, dass der Name mit der aktuellen Zählervariable n-mal ausgegeben
wird. Der Name und die Anzahl an Ausgaben wird wieder per Funktion als Parameter
übergeben. Die Ausgabe in der Konsole soll wie folgt aussehen:

\begin{lstlisting}
    Wie ist dein Name? Andreas
    Wie oft soll der Name ausgegeben werden? 5
    Andi 0
    Andi 1
    Andi 2
    Andi 3
    Andi 4
\end{lstlisting}

Falls die Aufgaben reibungslos funktioniert haben, kann der Studierende
die Herausforderung einer optionalen Bonusaufgabe annehmen. Um diese
Bonusaufgabe zu lösen, muss der Teilnehmer die vorherige Aufgabe modifizieren,
dass die Ausgabe in der Konsole wie folgt aussieht:

\begin{lstlisting}
    Wie ist dein Name? Andreas
    Wie oft soll der Name ausgegeben werden? 5
    Andi 1
    Andi 2
    Andi 3
    Andi 4
    Andi 5
\end{lstlisting}

Der Unterschied hier ist die Nummerierung. Im Standardfall zählt das Programm
von 0 bis n - 1. Um die Bonusaufgabe zu bewältigen, muss das Programm von 1 bis n
zählen.

% TODO: Super Mario Quelle
Die vierte und letzte Aufgabe beschäftigt sich mit der Generierung von
Text-Pyramiden, welche der ursprünglichen Version des Videospiels Super Mario
Brothers entsprechen sollen. Der Nutzer wird nach Start des Programms nach der gewünschten Größe bzw. Höhe der Pyramide. Nach Eingabe eines gültigen Werts
liefert das Skript, ausgegeben mit Raute-Zeichen, die linke Seite einer Super
Mario Pyramide. Die Ausgabe der gelösten Aufgabe sieht beispielhaft so aus:

% TODO: Höhe geht irgendwie nicht
\begin{lstlisting}
    Height: -1
    Height: 0
    Height: 6
         #
        ##
       ###
      ####
     #####
    ######
\end{lstlisting}

Das Beispiel zeigt neben der Pyramide auch, dass das Skript ungültige
Größenangaben erkennen und ignorieren soll. Aufgrund der erhöhten Schwierigkeit
dieser Aufgabe wurde entschieden, die letzte Aufgabe für die Studie zu
verwerfen.

\subsubsection{Test-Classroom}
Für die Vorbereitung der Studie wird eine neue GitHub-Organisation, sowie ein
neuer GitHub-Classroom angelegt. Sowohl das Replit-Template als auch die im
vorherigen Kapitel beschriebenen Aufgaben werden als Template-Repositorys in
der Test-Organisation abgelegt.

Für die Aufgaben werden mithilfe des Test-Frameworks pytest Benotungstests
programmiert und in GitHub-Classroom konfiguriert. Jede gelöste Aufgabe belohnt
den Studienteilnehmer pauschal mit 10 Punkten.

Neben den Programmieraufgaben werden auch ausführliche Anleitungen mit
Bilder geschrieben. Nach der Erstellung von Vorschaubildern können diese in der,
auch später produktiv genutzten, Plattform Tutors hochgeladen werden. Die
Studienteilnehmer erhalten zu Beginn des Probelaufs einen Link zur Übersicht der
benötigten Anleitungen in Tutors.

\subsection{Deskriptive Ergebnisse}
\begin{table}[H]
    \renewcommand*{\arraystretch}{1.6}
    \centering
    \begin{tabular}{|l|l|l|l|l|l|} 
    \hline
    \diagbox{\textbf{Fragen}}{\textbf{Ergebnisse}} & \textbf{Durchschnitt } & \textbf{SD} & \textbf{Median } & \textbf{Min.} & \textbf{Max.}  \\ 
    \hline
    \textbf{Alter }                                & 22.200                 & 0.837       & 22.000           & 21            & 23             \\ 
    \hline
    \textbf{PRE1 }                                 & 3.600                  & 2.300       & 3.000            & 1             & 6              \\ 
    \hline
    \textbf{PRE2 }                                 & 2.600                  & 1.670       & 3.000            & 1             & 5              \\ 
    \hline
    \textbf{PRE3 }                                 & 2.600                  & 0.894       & 2.000            & 2             & 4              \\ 
    \hline
    \textbf{PRE4 }                                 & 2.600                  & 1.820       & 2.000            & 1             & 5              \\ 
    \hline
    \textbf{PAST1 }                                & 2.600                  & 1.520       & 2.000            & 1             & 5              \\ 
    \hline
    \textbf{PAST2 }                                & 3.200                  & 1.300       & 4.000            & 1             & 4              \\ 
    \hline
    \textbf{PAST3 }                                & 3.800                  & 1.790       & 5.000            & 1             & 5              \\ 
    \hline
    \textbf{PAST4 }                                & 2.200                  & 1.100       & 3.000            & 1             & 3              \\ 
    \hline
    \textbf{PAST5 }                                & 3.800                  & 2.170       & 3.000            & 1             & 6              \\
    \hline
    \end{tabular}
\end{table}