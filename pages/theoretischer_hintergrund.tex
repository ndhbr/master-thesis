\section{Theoretischer Hintergrund und vorhandene Studiengangsfinder-Konzepte}
\subsection{Bisheriges Vorgehen bei der Studienorientierung}
% Wahl des Studiengangs PDF Seite 355
% Hochschulinformationstage wurden von MINT-Studierenden insgesamt häufig zur Studienentscheidung und Studienplanung genutzt (72 %).

% 1. Wichtigster Faktor Fach, 2. Ort o. Hochschultyp
% Oft kennen ein Fach, danach suchen sie einen Ort bzw. eine Hochschule
% Wenn sie dann auf die OTH Seite kommen, können sie nach der Kategorie suchen
% und dann ähnliche vielleicht noch besser passende Studiengänge finden
% Wichtigster Fachwahlgrund: Begabungen und pers. Neigungen
% Was Eltern, Verwandte u. Freunde sagen nur 0,1 %
% (Seite 67 von neuem PDF)
% 5 Typen schon Entscheidern, jeder entscheidet nach Pers. Neigung -> umso
% wichtiger ist es ihnen alle ähnlichen Studiengänge zu präsentieren!

Die Studienorientierung ist ein entscheidender Schritt im Bildungsweg eines
jeden Studieninteressierten und die Entscheidung für ein zukünftiges Studienfach
und eine Hochschule wird traditionell von verschiedenen Faktoren beeinflusst.
Eine umfassende Studie von CHE und EINSTIEG aus dem Jahr 2007 gibt Einblicke in
die Präferenzen und Entscheidungsmuster von Studieninteressierten.

Die Studie zeigt, dass für einen Großteil der Befragten (87,2 \%) die Wahl der
Hochschule und des Hochschulortes weniger relevant ist als das gewählte
Studienfach \parencite{einflussfaktoren}. Dies deutet darauf hin, dass das
persönliche Interesse an einem bestimmten Studienfach die Entscheidung stärker
beeinflusst als die geografische Lage oder der Ruf der Hochschule.

Ein weiteres wichtiges Ergebnis der Studie ist, dass 64,6 \% der Befragten ihr
Studienfach nach ihren Neigungen und Begabungen wählen
\parencite{einflussfaktoren}. Dies verdeutlicht, dass persönliche Neigungen und
individuelle Fähigkeiten die Studienentscheidung maßgeblich beeinflussen.

Die Erkenntnis, dass sich Studieninteressierte bei ihrer Studienfachwahl von
unterschiedlichen Motivationsfaktoren leiten lassen, führt zu der Vorstellung,
dass es verschiedene Entscheidungstypen gibt. Intrinsische Altruisten,
heimatgebundene Hedonisten, serviceorientierte Unabhängige und leistungsstarke
Karriereorientierte sind einige der identifizierten Gruppen, die jeweils
unterschiedliche Prioritäten bei der Fach-, Hochschul- und Ortswahl setzen.
Diese unterschiedlichen Entscheidungsmuster unterstreichen die Vielfalt
individueller Studienmotivationen. \parencite{einflussfaktoren}

In der Praxis bedeutet dies, dass Hochschulen wie die OTH-Regensburg darauf
achten müssen, ihre Studiengänge ansprechend und überzeugend zu präsentieren.
Gerade dann, wenn die Hochschule als mögliche Option in Betracht gezogen wird,
spielt die Möglichkeit, das gewünschte Fach entsprechend der individuellen
Interessen darzustellen, eine entscheidende Rolle.

Ein innovativer Studiengangsfinder, der Fächer inhaltlich kategorisiert und
ähnliche Studiengänge präsentiert, könnte hier eine wichtige Rolle spielen. Mit
einer solchen Lösung könnten Studieninteressierte effektiv über Alternativen
informiert werden, insbesondere wenn der ursprünglich angestrebte Studiengang
nicht verfügbar ist. Dies trägt dazu bei, dass die Hochschule potenzielle
Studierende auch dann überzeugen kann, wenn ihre erste Wahl nicht direkt
verfügbar ist, aber dennoch ähnliche, attraktive Alternativen bietet.


\subsection{Konzepte von Studiengangsfindern}
Die Historie der Studiengangsfinder zeigt eine dominierende Tendenz hin zu
umfragebasierten Konzepten. In diesem Zusammenhang werden Studierende durch
Umfragen zu ihren Interessen, Fähigkeiten und Präferenzen befragt, um auf dieser
Grundlage Studiengangsempfehlungen zu generieren.

Umfragebasierte Studiengangsfinder vertrauen auf die subjektiven Bewertungen
der Nutzer und versuchen, durch direkte Befragungen der Studierenden ihre
Präferenzen zu ermitteln. Die daraus resultierenden Empfehlungen basieren auf
den angegebenen Interessen und Vorlieben. Jedoch sind diese Empfehlungen stark
von der Qualität der gestellten Fragen und der Interpretation der Antworten
abhängig. Um die bewusste Lenkung des Algorithmus zu verhindern, enthalten viele
Umfrage-Tools ein Minimum von 50 Fragen. Die Umfrage enthält außerdem neben
einfachen Fragen wie \glqq Interessierst du dich für Informatik?\grqq{}, sehr
allgemein gehaltene (oft private) Fragen. Intelligente Algorithmen werten dann
die Antworten auf Aussagen, wie beispielsweise \glqq Wenn ich zur Party gehe,
suche ich Kontakt mit nur wenigen, die ich kenne.\grqq{} aus und versuchen durch
Zuordnung der Charakterzüge an bestimmte Interessensgruppen, passende
Studiengänge zu finden.

Trotz ihrer weiten Verbreitung weisen umfragebasierte Ansätze gewisse
Limitationen auf. Die Ergebnisse können stark von der Selbsteinschätzung der
Studierenden beeinflusst sein, und es besteht das Risiko von Verzerrungen oder
unvollständigen Informationen.


\subsection{Theorie und Anwendung von interaktiven ähnlichen Systemen}
