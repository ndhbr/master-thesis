\section{Theoretischer Hintergrund und vorhandene Studiengangsfinder-Konzepte}
\subsection{Bisheriges Vorgehen bei der Studienorientierung}

\subsection{Konzepte von Studiengangsfindern}
Die Historie der Studiengangsfinder zeigt eine dominierende Tendenz hin zu
umfragebasierten Konzepten. In diesem Zusammenhang werden Studierende durch
Umfragen zu ihren Interessen, Fähigkeiten und Präferenzen befragt, um auf dieser
Grundlage Studiengangsempfehlungen zu generieren.

Umfragebasierte Studiengangsfinder vertrauen auf die subjektiven Bewertungen
der Nutzer und versuchen, durch direkte Befragungen der Studierenden ihre
Präferenzen zu ermitteln. Die daraus resultierenden Empfehlungen basieren auf
den angegebenen Interessen und Vorlieben. Jedoch sind diese Empfehlungen stark
von der Qualität der gestellten Fragen und der Interpretation der Antworten
abhängig. Um die bewusste Lenkung des Algorithmus zu verhindern enthalten viele
Umfrage-Tools ein Minimum von 50 Fragen. Die Umfrage enthält außerdem neben
einfachen Fragen wie \glqq Interessierst du dich für Informatik?\grqq{}, sehr
allgemein gehaltene (oft private) Fragen. Intelligente Algorithmen werten dann
die Antworten auf Aussagen, wie beispielsweise \glqq Wenn ich zur Party gehe,
suche ich Kontakt mit nur wenigen, die ich kenne.\grqq{} aus und versuchen durch
Zuordnung der Charakterzüge an bestimmte Interessensgruppen, passende
Studiengänge zu finden.

Trotz ihrer weiten Verbreitung weisen umfragebasierte Ansätze gewisse
Limitationen auf. Die Ergebnisse können stark von der Selbsteinschätzung der
Studierenden beeinflusst sein, und es besteht das Risiko von Verzerrungen oder
unvollständigen Informationen.


\subsection{Theorie und Anwendung von interaktiven ähnlichen Systemen}
