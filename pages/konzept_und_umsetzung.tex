\section{Konzept und Umsetzung an der OTH-Regensburg}
\subsection{Vorstellung der OTH-Regensburg als Fallbeispiel}
Die Ostbayerische Technische Hochschule Regensburg (OTH-Regensburg) dient in
dieser Masterarbeit als praxisorientiertes Fallbeispiel für die Konzeption und 
Umsetzung eines innovativen Studiengangsfinders.

\subsubsection{Über die OTH-Regensburg}
Die OTH-Regensburg, gegründet im Jahr 1971, ist eine der größten Hochschulen
für angewandte Wissenschaften in Deutschland. Mit ihrem breiten Spektrum an 
praxisorientierten Studiengängen und ihrer engen Verbindung zur Industrie bietet
die OTH-Regensburg eine ideale Umgebung für die Entwicklung und Implementierung
eines innovativen Studiengangsfinders. \parencite{ueber-die-oth}

Die Hochschule zeichnet sich durch eine moderne Infrastruktur, hochqualifizierte 
Dozenten und eine enge Zusammenarbeit mit regionalen Unternehmen aus. Mit mehr
als 50 Bachelor- und Masterstudiengängen in den Bereichen
Ingenieurwissenschaften, Naturwissenschaften, Wirtschaft und
Sozialwissenschaften bietet die OTH-Regensburg eine breite Palette an 
Studienmöglichkeiten. \parencite{ueber-die-oth}

\subsubsection{Herausforderung und Zielsetzung}
Wie viele Bildungseinrichtungen steht auch die OTH-Regensburg vor der
Herausforderung, ihre Vielzahl an Studiengängen für Studieninteressierte
transparenter und zugänglicher zu machen. Der Studiengangsfinder, der im Rahmen
dieser Masterarbeit entwickelt wird, soll dazu beitragen, potenziellen
Studierenden eine klarere Orientierung über die verfügbaren Studienmöglichkeiten
zu bieten und ihre Entscheidungsfindung zu unterstützen.

Durch die Anwendung des Studiengangsfinders an der OTH-Regensburg wird nicht nur
ein innovatives Werkzeug für Studieninteressierte geschaffen, sondern auch die 
Effektivität und Transparenz der Studienorientierung an der Hochschule selbst 
verbessert. Die Integration und algorithmische Verarbeitung von Daten zu 
Studiengängen und Schwerpunkten ermöglicht eine präzisere Darstellung der 
Studienvielfalt und fördert gleichzeitig die strategische Ausrichtung der
Hochschule.

\subsection{Konzept für eine automatisch generierte Infografik}

\subsection{User-centered Design Research: Mockup eines Prototypen}

\subsection{Beschreibung der Positionsberechnung der Infografik}

\subsection{Technische Details zur Umsetzung}

\subsection{Anwendung des entwickelten Systems auf die Studiengänge der OTH-Regensburg}